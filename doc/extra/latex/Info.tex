\chapter{Getting/setting informations}

Several functions have been implemented in Move3D in order to get some
informations on the environment or to set some values and parameters.

\section{Selecting the current element}

These functions are used to select the current element of a scene
(environment, obstacle, robot, trajectory) or to get information on
this current element.

The function {\tt int p3d\_get\_desc\_number(int type)}
\index{p3d\_get\_desc\_number} returns the
number of elements of type {\tt type} ({\tt P3D\_ENV,
P3D\_OBSTACLE, P3D\_ROBOT, P3D\_BODY, P3D\_TRAJ}).

The functions {\tt char *p3d\_get\_desc\_curname(int type)}
\index{p3d\_get\_desc\_curname}, {\tt void *p3d\_get\_desc\_curid (int
type)} \index{p3d\_get\_desc\_curid} and {\tt int p3d\_get\_desc\_curnum(int
type)} \index{p3d\_get\_desc\_curnum} return the name, pointer and
number of the current element of type {\tt type}.

The current element can been changed by selecting another element of
the description.  The functions {\tt void *p3d\_sel\_desc\_name(int
type, char* name)} \index{p3d\_sel\_desc\_name}, {\tt void
*p3d\_sel\_desc\_num(int type, int num)} \index{p3d\_sel\_desc\_num}
and {\tt void *p3d\_sel\_desc\_id(int type, void *id)}
\index{p3d\_sel\_desc\_id} select the
element of type {\tt type}, of name {\tt name}, of number {\tt num} or
of pointer {\tt id}, and returns this element.

\section{Environments, polyhedrons and obstacles}

\subsection{Environments}

The user can get informations about the bounding box of the
current environment by calling the function {\tt void
p3d\_get\_env\_box(double *x1, double *x2, double *y1, double *y2,
double *z1, double *z2)} \index{p3d\_get\_env\_box}, or modify it by
calling the function {\tt int  p3d\_set\_env\_box(double x1, double
x2, double y1, double y2, double z1, double z2)}
\index{p3d\_set\_env\_box}.  This last function is usually used to
complete the description of a scene. 

The function {\tt void p3d\_env\_info(void)} \index{p3d\_env\_info} prints some informations
about the current environment : name, id, number of the environment in
the Move3D session, number of obstacles, of robots, and informations
about all the obstacles and robots.

\subsection{Polyhedrons}

The function {\tt int p3d\_poly\_get\_nb(void)}
\index{p3d\_poly\_get\_nb} returns the number of polyhedrons of a
scene. 

The function {\tt p3d\_poly *p3d\_poly\_get\_poly(int i)}
\index{p3d\_poly\_get\_poly} returns the i-th polyhedron of the
scene. The function {\tt p3d\_poly
*p3d\_poly\_get\_poly\_by\_name(char *name)}
\index{p3d\_poly\_get\_poly\_by\_name} returns the polyhedron of
the scene of name {\tt name}. The function {\tt p3d\_poly
*p3d\_poly\_get\_first (void)} \index{p3d\_poly\_get\_first}
initializes a pointer on the begining of the polyhedrons list and
returns the first polyhedron of this list. The function {\tt p3d\_poly
*p3d\_poly\_get\_next(void)} \index{p3d\_poly\_get\_next} returns
the next polyhedron of this list.

The function {\tt void p3d\_get\_info\_pos\_poly(p3d\_poly *p)}
\index{p3d\_get\_info\_pos\_poly} prints
the position matrix of the polyhedron {\tt p}.

The function {\tt void p3d\_BB\_get\_BB\_poly1(p3d\_poly *p, double
*x1, double *x2, double *y1, double *y2, double *z1, double *z2)}
\index{p3d\_BB\_get\_BB\_poly1} returns
the bounding box of the polyhedron {\tt p}.

\subsection{Obstacles}

%move\_point\\
The function {\tt p3d\_obj *p3d\_get\_obst\_by\_name(char *name)}
\index{p3d\_get\_obst\_by\_name} returns the object of the scene of
name {\tt name}.

The function {\tt int p3d\_get\_obstacle\_npoly(void)}
\index{p3d\_get\_obstacle\_npoly} returns the
number of polyhedrons of the current obstacle. The function {\tt int
p3d\_get\_obstacle\_npt(int i)} \index{p3d\_get\_obstacle\_npt}
returns the number of vertices of the i-th polyhedron of the current
obstacle. The function {\tt void p3d\_get\_obstacle\_pt(int num,int i,
double *x, double *y, double *z)} \index{p3d\_get\_obstacle\_pt}
returns the coordinates {\tt (x,y,z)} of the i-th vertex of the
polyhedron of number {\tt num} of the current obstacle. 

The function {\tt void p3d\_get\_BB\_obj(p3d\_obj *o,double *x1,double
*x2,double *y1,double *y2,double *z1,double *z2)}
\index{p3d\_get\_BB\_obj} returns the bounding box of the object {\tt o}.

The function {\tt void p3d\_obstacle\_info(void)}
\index{p3d\_obstacle\_info} prints informations
about the current obstacle : name, number, id, bounding box,
polyhedrons,...

\section{Robots}

\subsection{Robot}

The value of some parameters of the current robot can be modified
after its description. 

The function {\tt int p3d\_set\_robot\_box(double x1, double x2,
double y1, double y2, double z1, double z2, double t1, double t2)}
\index{p3d\_set\_robot\_box} set
the values of the box of the current robot, i.e. the bounds of the
degrees of freedom of the first joint j0, $x,y,z,\theta$. This
function is usually used to complete the description of a scene. 
The function {\tt void p3d\_set\_robot\_radius(double radius)}
\index{p3d\_set\_robot\_radius} sets
the value of the radius of the turning circle of the current robot. 
The function {void p3d\_set\_ROBOT\_GOTO(double *q)}
\index{p3d\_set\_ROBOT\_GOTO} set the value of
the goal configuration of the current robot.

Some general informations about the robot can also be get.

The function {\tt double p3d\_get\_robot\_radius(void)}
\index{p3d\_get\_robot\_radius} gets the
radius of the turning circle of the current robot. The function {\tt
void p3d\_get\_robot\_box(double *x1, double *x2, double *y1, double
*y2,double *z1,double *z2,double *t1,double *t2)}
\index{p3d\_get\_robot\_box} returns the values of
the box of the current robot.

The function {\tt void p3d\_get\_BB\_rob(p3d\_rob *r,double *x1,double
*x2,double *y1,double *y2,double *z1,double *z2)}
\index{p3d\_get\_BB\_rob} returns the bounding
box of the robot {\tt r}.

The function {\tt void p3d\_robot\_info(void)}
\index{p3d\_robot\_info} prints informations
about the current obstacle : name, number, id, number of bodies,
number of joints, bounding box, position, bodies, polyhedrons,...

\subsection{Body}

The function {\tt p3d\_obj *p3d\_get\_body\_by\_name(char *name)}
\index{p3d\_get\_body\_by\_name} 
returns the body of the current robot of name {\tt name}.

The function {\tt int p3d\_get\_body\_npoly(void)}
\index{p3d\_get\_body\_npoly} returns the
number of polyhedrons of the current body of the current robot. The function {\tt int
p3d\_get\_body\_npt(int i)} \index{p3d\_get\_body\_npt} returns the number of vertices of the
i-th polyhedron of the current body of the current robot. The function {\tt void
p3d\_get\_body\_pt(int num, int i, double *x, double *y, double *z)} \index{p3d\_get\_body\_pt}
returns the coordinates {\tt (x,y,z)} of the i-th vertex of the
polyhedron of number {\tt num} of the current body of the current robote. 

The function {\tt void p3d\_get\_BB\_obj(p3d\_obj *o,double *x1,double
*x2,double *y1,double *y2,double *z1,double *z2)}
\index{p3d\_get\_BB\_obj} can also be used to
get the bounding box of the body {\tt o}.

The function {\tt void p3d\_obstacle\_info(void)}
\index{p3d\_obstacle\_info} prints informations
about the current obstacle : name, number, id, bounding box,
polyhedrons, joint corresponding to this body...

\subsection{Joints}

The function {\tt int p3d\_get\_robot\_njnt(void)}
\index{p3d\_get\_robot\_njnt} returns the number
of joints of the current robot.

The function {\tt int p3d\_get\_robot\_jnt\_type(int i)}
\index{p3d\_get\_robot\_jnt\_type} returns the type ({\tt P3D\_ROTATE}
or {\tt P3D\_TRANSLATE}) of the i-th joint of the current robot. The
function {\tt void p3d\_get\_robot \_jnt\_bounds(int i, double *vmin,
double *vmax)} \index{p3d\_get\_robot\_jnt\_bounds} returns the
minimum and maximum bounds of the i-th joint of the current robot.

\subsection{position of a robot}

The kinematic structure of a Move3D robot is a tree who root is the
specific joint j0 (cf Chapter \ref{model}). This joint allows the
robot to translate along the $x$, $y$ and $z$ axis and to rotate
around the $z$ axis. It correspond, for example, to the four degrees
of freedom of a plane. Every other joint adds a degree of freedom in
rotation or translation. The different steps for setting a new
position for the current robot come from this structure.


The function {\tt void p3d\_set\_robot\_pos(double x, double y, double
z, double t)} \index{p3d\_set\_robot\_pos} changes the values of the
degrees of freedom of the joint j0 of the current robot. The function
{\tt void p3d\_set\_robot\_jnt(int i, double v)}
\index{p3d\_set\_robot\_jnt} changes the current value of the ith
joint of the current robot.

\begin{figure}[hbt]
\centerline{
\psfig{figure=FIG/setpos2.ps,width=14cm}
}
\caption{\small 
Setting the position of the current robot.
}
\label{FIG_SETPOS1}
\end{figure}

Once the user has changed the values of all or some of the joints, the
position of the robot has been changed but not updated. The function
{\tt int p3d\_update\_robot\_pos(void)}
\index{p3d\_update\_robot\_pos} must be called. This function
computes the new position matrix of all the joints that have been
affected by the changes (the joints whose value has been changed and
the joints placed after those joints in the cinematic chain). It also
updates the position of the robot's bodies, and the boundings boxes of
the robot, objects and polyhedrons.
%{\tt
%p3d\_BB\_update\_BB\_obj1\\
%p3d\_BB\_update\_BB\_obj2\\
%p3d\_BB\_update\_BB\_rob\\
%p3d\_col\_set\_pos\\
%}

The position of a robot can be get using the following functions. The
function {\tt void p3d\_get\_robot\_pos(double *x, double *y, double
*z, double *t)} \index{p3d\_get\_robot\_pos} returns the values of the
degrees of freedom of the first joint j0 of the current robot,
$x,y,z,\theta$. The function {\tt void p3d\_get\_robot\_jnt(int i,
double *val)} \index{p3d\_get\_robot\_jnt} returns the value of
the i-th joint of the current robot.

Two functions allow the user to evaluate the distance between two
configurations of the current robot.

The function {\tt double p3d\_dist(p3d\_rob *r, p3d\_courbe *c, int
nrs)} \index{p3d\_dist} returns the distance covered by the robot {\tt r} along the path
{\tt c}. The function {\tt double p3d\_dist\_q1\_q2 (p3d\_rob *r,
double *q1, double *q2)} \index{p3d\_dist\_q1\_q2} returns the distance covered by the robot
{\tt r} between the configuration {\tt q1} and {\tt q2} with the
current local method. The distance are computed the same way : to the
lenght of the curve covered by the four degrees of freedom of j0 is
added the square root of the sum of the difference of all the degrees
of freedom that are not $x,y,z$. This way the distance between two
configurations can be evaluated whatever local method and
configurations have been chosen.

\section{Changing a color}

Others commands exist that allow the user to describe more complex
scenes.

When they are created, the default color of the obstacles is blue,
and the default color of the robots is yellow. But we can use other
colors to make the scene more realistic.

The function {\tt void p3d\_set\_obst\_color(char *name, int color)}
\index{p3d\_set\_obst\_color} with the name {\tt obst} and the code
number of the color {\tt color} as input sets the color of all the
polyhedrons of the obstacle {\tt obst} to {\tt color}. The function
{\tt void p3d\_set\_obst\_poly\_color(char *name,int num, int color)}
\index{p3d\_set\_obst\_poly\_color} does exactly the same for the ith
polyhedron of the obstacle {\tt obst}.

The functions {\tt void p3d\_set\_body\_color(char *name, int color)}
\index{p3d\_set\_body\_color} and {\tt void
p3d \_set\_body\_poly\_color(char *name, int num, int color)}
\index{p3d\_set\_body\_poly\_color} do exactly the same.

Changing a color can only be done when the description of the obstacle 
or the description of the robot (for the bodies) have been completed.

The colors available in Move3D are White, Black, Blue, Red, Yellow,
Green, Grey and Brown.

\section{Graphs and trajectories}

The user can also get some information on the roadmaps built by Move3D (cf Chapter
\ref{global}). 

The function {\tt void p3d\_print\_info\_graph(p3d\_graph *G)}
\index{p3d\_print\_info\_graph} prints
some info about the graph {\tt G} : number of nodes, number of
configuration generated, number of free configurations obtained, time
needed to build this graph, calls to the collision checker, to the
bounding box test, to the local method.

The function {\tt void p3d\_print\_graph(p3d\_graph *G)}
\index{p3d\_print\_graph} printf the
number of node of the graphe {\tt G}, the number of connex composants
of this graphe and, for each connex composant, the number of the
composant, the number of nodes of this composant, the number of each node of this
composant and the number of neighbours of this node.

The user can also get some informations about the current  trajectory
of the current robot.

The function {\tt int p3d\_get\_traj\_ncourbes(void)}
\index{p3d\_get\_traj\_ncourbes} returns the
number of elementary curves contained in the current trajectory.
The function {\tt voit p3d\_get\_traj\_pos(int i, double *q)}
\index{p3d\_get\_traj\_pos} returns the i-th configuration on the
curve regarding the current discretization step.


